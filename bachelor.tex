\documentclass[11pt, a4paper, german]{article}
\usepackage{amsmath,amsthm,amssymb,latexsym,amsfonts}
\usepackage[left=3cm,right=3cm,top=2	cm,bottom=2cm]{geometry} % page settings
\setlength{\parindent}{0mm}

\usepackage[german]{babel}
\usepackage{BA_Titelseite}

\newtheorem{theorem}{Theorem}
\newtheorem{definition}[theorem]{Definition}
\newtheorem{example}[theorem]{Example}
\newtheorem{remark}[theorem]{Remark}

\numberwithin{equation}{section}
\numberwithin{theorem}{section}

%Namen des Verfassers der Arbeit
\author{Konstantins Starovoitovs}
%Geburtsdatum des Verfassers
\geburtsdatum{12. Oktober 1992}
%Gebortsort des Verfassers
\geburtsort{Riga, Lettland}
%Datum der Abgabe der Arbeit
\date{\today}

%Name des Betreuers
% z.B.: Prof. Dr. Peter Koepke
\betreuer{Betreuer: Prof. Dr. Margherita Disertori}
%Name des Instituts an dem der Betreuer der Arbeit tätig ist.
\zweitgutachter{Zweitgutachter: Prof. Dr. X Y}
%z.B.: Mathematisches Institut
\institut{Institute for Applied Mathematics}
%Titel der Bachelorarbeit
\title{Extensions of symmetric operators in context of particle interaction}
%Do not change!
\ausarbeitungstyp{Bachelorarbeit Mathematik}

\DeclareMathOperator{\Exists}{\exists}
\DeclareMathOperator{\Forall}{\forall}

\begin{document}

\maketitle

%\clearpage

\section{Particle kinematics}

Before we dive into theory of symmetric operators and their extensions, we want to describe model of particle physics based on \cite{griffiths}.\\

Unlike in Newtonian mechanics, where state of the system is given by the vectors in Euclidean space $\mathbb{R} ^d$ and physical quantities by functions on this space, states in quantum mechanics are represented by functions in a Hilbert space $\mathcal{H} = L_2(\mathbb{R} ^d)$ and physical quantities by self-adjoint operators on this space. The functions $\Psi \in \mathcal{H}$ are called wave functions.\\

In fact functions that are multiples of one another represent the same quantum state (so basically we could say that the state is represented by an equivalence class of wave functions). It is customary to normalize wave functions so that $\int |\Psi|^2 d^3x=1$. Then wave functions stand for probability amplitude of the state (meaning that $|\Psi |^2$ is probability distribution, i.e. $|\Psi(x,y,z,t)|^2 \, dx \, dy\,dz$ is the probability of finding the the particle in the volume element $dx\,dy\,dz$).\\

@todo: position, momentum, mass, relativistic adjustments (Lorentz transformation), energy, conservation laws.\\

Quantum mechanics postulates that orbital angular momentum $L$ and spin angular momentum $S$ of a particle are given by the formulas:
\begin{align}
L^2 &= l(l+1)\hbar^2 &l   &\in \mathbb \{0, 1, 2, \dots\} \label{eq:l2} \\
L_z &= m_l\hbar      &m_l &\in \{-l, -l+1, \dots, l-1, l\} \label{eq:lz}\\
S^2 &= s(s+1)\hbar^2 &s   &\in {0, 1/2, 1, 3/2, \dots} \label{eq:s2}\\
S_z &= m_s\hbar      &m_s &\in \{-s, -s+1, \dots, s-1, s\} \label{eq:sz}
\end{align}

Notice that we can only measure one coordinate of momentum ($z$ by convention) and its modulus squared due to Heisenberg's uncertainty principle.

\begin{definition}
Particles of half-integer spin are called \textbf{fermions}, particle of integer spin are called \textbf{bosons}.
\end{definition}

An important difference between fermions and bosons is that two identical bosons can occupy the same quantum space, whereas fermions cannot (Pauli exclusion principle). This accounts for the fact that two identical bosons have symmetrical wave function ($\psi(x, y) = \psi(y, x)$) and fermions have antisymmetrical wave function ($\psi(x, y) = -\psi(y, x)$). Therefore, if we assume for two fermionic particles that $x=y$, then $\psi(x,y)=\psi(y,x)=0$, which is consistent with Pauli exclusion principle.\\

\subsection{One particle system in the presence of central potential}

The principal foundation for non-relativistic quantum mechanics is Schrödinger equation. TThe law of conservation of energy in quantum mechanics signifies that if in a given state the energy (sum of the kinetic and potential energy) has a definite value, this value remains constant in time:

\begin{equation} \label{eq:energy}
\frac{1}{2m} p^2 + V = E
\end{equation}

State in which the energy has definite values are called \textbf{stationary states} of a system. The stationary state with the smallest possible value of the energy is called \textbf{ground state}.

In quantum mechanics the momentum operator is $p = i\hbar\nabla$ and the energy operator is $E = i\hbar \frac{\partial}{\partial t}$.\\

So the expression above can be rewritten as 

\begin{equation} \label{eq:schroedinger}
\left(- \frac{\hbar^2}{2m}\Delta + V\right)\Psi = i\hbar \frac{\partial}{\partial t} \Psi
\end{equation}

where $\Psi(x,y,z,t)$ is our wave function.\\

Notice that the Schrödinger equation applies for the cases when the constituents of the system travel at speeds substantially less than $c$ (hydrogen, hadrons made out of heavy quarks and others, as opposed to, for example, photon), so we can use non-relativistic quantum mechanics.\\

This is the time-dependent Schrödinger equation which describes the time-evolution of the system of one particle of mass $m$ in the presence of a specified potential energy $V$.\\

In case $\Psi$ is self-adjoint, we can apply spectral theorem and can write

\begin{equation} \label{eq:wave-function-separation}
\Psi(x,y,z,t) = \psi(x,y,z) f(t)
\end{equation}

By separation of variables we get 

\begin{equation} \label{eq:schroedinger-after-separation}
\frac{1}{\psi}\left(- \frac{\hbar^2}{2m}\Delta + V\right)\psi = \frac{i\hbar}{f}\frac{df}{dt}
\end{equation}

The only way this equation can hold for all $x,y,z$ and $t$ is if both sides are constant:

\begin{align} 
\left(- \frac{\hbar^2}{2m}\Delta + V\right)\psi = E \psi \label{eq:psi}\\
i\hbar \left(\cfrac{\partial f}{\partial t} \right) = Ef \label{eq:f}
\end{align}

We can easily solve (\ref{eq:f}) and get the following formula for the time-evolution of the energy of the system: 

\begin{equation} \label{eq:time-evolution}
f(t) = \exp(-itE/\hbar)
\end{equation}

The operator $H:= - \frac{\hbar^2}{2m}\Delta + V$ is called \textbf{Hamiltonian} and corresponds to the total energy of the system. (\ref{eq:psi}) has eigenvalue form: 

\begin{equation} \label{eq:schroedinger-eigenvalue-form}
H\psi = E\psi
\end{equation}

So after we find $\psi$ which satisfies the first equation, the complete wave function for a particle of mass $m$ and energy $E$, under the influence of potential energy $V(x,y,z)$ is 

\begin{equation}
\Psi(x,y,z,t) = \psi(x,y,z)\exp(-iEt/\hbar)
\end{equation}

It is often the case that the potential $V$ is spherically symmetrical, i.e. depends only on distance from origin $r$ (for example, Coulomb's law).

In this case we can use usual spherical coordinates, in which Laplacian taken the form:

$$ \Delta = \frac{1}{r^2} \frac{\partial}{\partial r}\left(r^2 \frac{\partial}{\partial r}\right) + \frac{1}{r^2 \sin \theta} \frac{\partial}{\partial \theta} \left( \sin \theta \frac{\partial}{\partial \theta}\right) + \frac{1}{r^2 \sin^2 \theta}\frac{\partial ^2}{\partial \phi ^2}$$

By writing $\psi(r, \theta, \phi) = R(r)\Theta(\theta)\Phi(\phi)$, we acquire the following system of ordinary differential equations:

\begin{align} 
\frac{1}{r^2}\frac{d}{dr}\left(r^2 \frac{dR}{dr}\right) &= \left(\frac{l(l+1)}{r^2} + \frac{2m}{\hbar^2}(V(r)-E)\right) R \label{eq:R}\\
\sin \theta \frac{d}{d\theta}\left( \frac{d\Theta}{d \theta}\right) &= \left( m_l^2 - l(l+1)\sin^2\theta\right)\Theta \label{eq:Theta} \\
\frac{d^2\Phi}{d\phi^2} &= -m_l^2\Phi \label{eq:Phi}
\end{align}

where $l$ and $m_l$ are from (\ref{eq:l2}) and (\ref{eq:lz}).\\

It is customary that the solutions of (\ref{eq:Theta}) and (\ref{eq:Phi}) are combined and presented in the form of the spherical harmonics $Y^{m_l}_l(\theta, \phi)$. \\

The equation (\ref{eq:R}) can be simplified by setting $u(r) := r R(r)$ which transforms (\ref{eq:R}) into so-called radial Schrödinger equation:

\begin{equation} \label{eq:radial-schroedinger-equation}
-\frac{\hbar^2}{2m}\frac{d^2u}{dr^2}+\left(V(r)+\frac{\hbar^2}{2m}\frac{l(l+1)}{r^2}\right) = Eu
\end{equation}

At this point the strategy is to solve (\ref{eq:radial-schroedinger-equation}) for particular potential $V(r)$ and combine the result with appropriate spherical harmonic to get the full wave function.

@todo for most values of $E$ the solution blows up at large $r$ and yields a non-normalizable wave function. Such solution does not represent a possible physical state. So a bound system cannot have just any old energy, but can take energy on only certain specific values, the so-called 'allowed energies' of the system. Our real concern is not with the wave function itself, but with the spectrum of allowed energies.

\subsection{Two-particle system with zero-range interaction}

\begin{definition}
We say that particles $x_1$ and $x_2$ have \textbf{zero-range interaction} if 

\begin{equation}
V(x_1, x_2) = \delta_{x_1-x_2}
\end{equation}

\end{definition}

We described above the apparatus for one particle in the presence of central potential $V(r)$. However, the same model can be used to describe two-particle systems in the center of mass reference frame, if the difference in particle masses is overwhelming. For example, hydrogen atom consists of one proton and one electron which are interacting by electromagnetic force. The mass of the proton is so much larger that we can assume that proton basically lies at the origin and the potential in question is given by the Coulomb's law, so $V(r) \cong \frac 1r$.\\

If we are dealing though with two particles of comparable masses, this two-body problem can be converted into an equivalent one-body problem with the reduced mass $m_\text{red}$ in the center of mass reference frame:

\begin{align}
m_\text{red} &=\frac{m_1m_2}{m_1+m_2}\\
p&=p_1=-p_2\\
H &= \frac{p^2}{2m_\text{red}} + V(r)
\end{align}

\begin{equation} \label{eq:general-hamiltonian}
H = -\sum\Delta_i
\end{equation}

\section{Self-adjoint operators}

As mentioned before, physical quantities in quantum mechanics are given by self-adjoint operators. The model described in the previous chapter assumes existence of a well-defined Hamiltonian. However, the Hamiltonian defined in (\ref{eq:general-hamiltonian}) is not necessarily self-adjoint, so in order to get a well-defined problem we have to adjust the operator $H$. This is where extensions of symmetric operators come into play.\\

We are used to deal with well-behaved bounded operators (meaning $\Vert A\Vert = \sup_{\Vert \psi \Vert = 1} \Vert A \psi \Vert < \infty$). However, in physics we often encounter operators which are not bounded. Before we proceed, first we have to define how theory of bounded operators extrapolates to the unbounded operators. \\

We start with some Hilbert space $\mathcal{H}$.

\begin{definition} The linear operator $T$ acting on the space $\mathcal{H}$ is called \textbf{bounded}, if $$\sup_{\Vert x \Vert = 1}  \Vert Tx \Vert < \infty.$$ If a linear operator is not bounded, we call it \textbf{unbounded}. We denote with $D(T)$ the domain of the operator $T$.
\end{definition}

\begin{definition}
The \textbf{graph} of the linear operator $T$ is defined by $$\Gamma(T) := \{(\varphi, T\varphi): \varphi \in D(T)\}.$$ We consider the graph $\Gamma(T)$ as the subspace of the Hilbert space $\mathcal{H}\times \mathcal{H}$ equipped with the scalar product $$((\varphi_1, \psi_1), (\varphi_2, \psi_2))_{\mathcal{H}\times\mathcal{H}} := (\varphi_1, \varphi_2)_{\mathcal{H}} + (\psi_1, \psi_2)_{\mathcal{H}}.$$
\end{definition}

\begin{definition}
Let $T_1$ and $T_2$ be operators on $\mathcal{H}$. We say that $T_2$ is an \textbf{extension} of $T_1$, if $$\Gamma(T_1) \subset \Gamma(T_2).$$
\end{definition}

We observe that the definition implies that $D(T_1)\subset D(T_2)$ and $T_2|_{D(T_1)} = T_1$.

\begin{definition}
An operator $T$ is called \textbf{closable} if it has a closed extension. Every closable operator has the smallest closed extension, called its \textbf{closure}, which we denote by $\overline{T}$.
\end{definition}

\begin{definition}
Let $T$ be a densely defined linear operator on a Hilbert space $\mathcal{H}$. We denote with $$D(T^*) := \{\varphi \in \mathcal{H} : \Exists \eta \in \mathcal{H} \Forall \psi \in D(T) (T\psi, \varphi) = (\psi, \eta) \}.$$ For each such $\varphi \in D(T^*)$ we define $T^*\varphi := \eta$. The operator $T^*$ is called the \textbf{adjoint} of $T$.
\end{definition}

\begin{definition}
A densely defined linear operator $T$ on the Hilbert space $\mathcal{H}$ is called \textbf{symmetric} if $T\subset T^*$. Equivalently, $T$ is symmetric if and only if $$(T\varphi, \psi) = (\varphi, T\psi) \Forall \psi, \varphi \in D(T).$$
\end{definition}

\begin{definition}
$T$ is called \textbf{self-adjoint} if $T = T^*$, i.e. if and only if $T$ is symmetric and $D(T) = D(T^*)$. A symmetric operator $T$ is called \textbf{essentially self-adjoint} if $T^*$ is self-adjoint.
\end{definition}

Self-adjoint operators have several useful properties:

\begin{itemize}
\item Spectral theorem holds.
\item Spectrum is a subset of real line.
\item Self-adjoint operators can be exponentiated ($H\mapsto \exp(-itH)$).
\item compatible with Stone's theorem.
\end{itemize}

\begin{theorem} (Hellinger--Toeplitz)
Everywhere defined self-adjoint operator is bounded.
\end{theorem}

The fact that everywhere defined self-adjoint operators are bounded suggests that unbounded self-adjoint operators are defined not everywhere, but rather on a dense subspace of $H$.

\begin{example}

Let $d=1$ and $(T \varphi)(x) := x \varphi (x)$ (position operator). We notice that $T \varphi \in \mathcal{H}$ if and only if $\int x^2 |\varphi(x)|^2 < \infty$, so we define $D(T):= \{\varphi \in \mathcal{H} :\int x^2 |\varphi(x)|^2 < \infty \}$. We notice that the operator is unbounded (i.e. we can find a sequence $\Vert T \varphi_n \Vert \rightarrow \infty$ while keeping $\Vert \varphi_n \Vert = 1$, consider, for example, standard mollifiers). So $T$ is an example of an unbounded self-adjoint operator defined on a dense subset of $\mathcal{H}$.

\end{example}


\section{Skornyakov--Ter-Martirosyan extensions}

\section{Next}

First of all, we have to notice that factor of the Hilbert space that correspond to the fermions is $ L ^{2} _{\text{asym} } \left( \mathbb{R}^{6} \right) $, so that we get $ \psi \left( x _{1} , x _{2} \right) = - \psi \left( x _{2} , x _{1} \right) $. This yields that $ \psi \left( x_1, x_1 \right) =0 $, i.e. two particles can't be at the same place simultaneously, while still preserving modulus squared (accounting for the fact that the particles are indistinguishable).

Hamiltonian
$ H _{0} = - \frac{1}{2} \left( \frac{1}{m} \Delta _{y} + \Delta _{x _{1} } + \Delta _{x _{2} } \right) $ (why?).

After applying Fourier transform, we notice that:

$ \mathcal{F} H _{0} \mathcal{F} ^{-1} \left( \phi \left( q, k _{1} , k _{2} \right) \right) = \left( \frac{1}{m} \left| q \right| ^{2} + \left| k _{1} \right| ^{2} + \left| k _{2} \right| ^{2} \right) \left( \phi \left( q, k _{1} , k _{2} \right) \right) $.

Change of coordinates
After Fourier transformation (going to momentum space) we make the change of coordinates:

$ P = q+k _{1} + k _{2} ; \,\,\,\, p _{j} = \frac{P }{m+2} - k _{j} $

Corresponding matrices are:

$ U = \left( \begin{array}{ccc} 1 & 1 & 1 \\ \tfrac{1}{m+2} & \tfrac{1}{m+2} -1 & \tfrac{1}{m+2} \\ \tfrac{1}{m+2} & \tfrac{1}{m+2} & \tfrac{1}{m+2} -1 \end{array} \right) ; \,\,\,\, U ^{-1} = \left( \begin{array}{ccc} 1 - \tfrac{2}{m+2} & 1 & 1 \\ \tfrac{1}{m+2} & -1 & 0 \\ \tfrac{1}{m+2} & 0 & -1 \end{array} \right) $

So we get:

$ \left\{\begin{array}{rcl} q & = & (1 - \frac{2}{m+2} ) P + p _{1} + p _{2} \\ k _{1} & = & \frac{P}{m+2} - p _{1} \\ k _{2} &=& \frac{P}{m+2} - p _{2} \end{array}\right\} $

which yields:

$ \tfrac{1}{m} \left| q \right| ^{2} + \left| k _{1} \right| ^{2} + \left| k _{2} \right| ^{2} = \\ \tfrac{P ^{2} }{m} - \tfrac{4 P ^{2} }{m \left( m+2 \right) } + \tfrac{4 P ^{2} }{m \left( m+2 \right) ^2 } + \tfrac{2 P ^2 }{\left( m+2 \right) ^2 } + \tfrac{2 P p _{1} }{m} - \tfrac{4 P p _{1} }{m \left( m+2 \right) } - \tfrac{2 P p _{1} }{m+2} + \tfrac{2 P p _{2} }{m} - \tfrac{4 P p _{2} }{m \left( m+2 \right) } - \tfrac{2 P p _{2} }{m+2} + \tfrac{p _{1} ^2 }{m} + \tfrac{2 \left( p _{1} , p _{2} \right) }{m} + \tfrac{p _{2} ^2 }{m} + p _{1} ^2 + p _{2} ^2 = \\ \tfrac{P ^{2} }{m} - \tfrac{4 P ^{2} }{m \left( m+2 \right) } + \tfrac{4 P ^{2} }{m \left( m+2 \right) ^2 } + \tfrac{2 P ^2 }{\left( m+2 \right) ^2 } + \tfrac{p _{1} ^2 }{m} + \tfrac{2 \left( p _{1} , p _{2} \right) }{m} + \tfrac{p _{2} ^2 }{m} + p _{1} ^2 + p _{2} ^2 = \\ \frac{P ^{2} }{m+2} + \frac{m+1}{m} \left( p _{1} ^2 + p _{2} ^2 + \frac{2}{m+1} \left( p _{1} , p _{2} \right) \right) . $


Hence we can write $ \widetilde{H _{0} } = \mathcal{F} H _{0} \mathcal{F}^{-1} = \widetilde{H _{0} } ^{\left( 1 \right) } + \frac{m}{m+1} \widetilde{H _{0} } ^{\left( 2 \right) } $ with

$ \left( \widetilde{H _{0} } ^{\left( 1 \right) } f \right) \left( P \right) = \frac{P ^{2} }{m+2} f \left( P \right) $ for $ f \in L _{2} \left( \mathbb{R}^{3} \right) $

and

$ \widetilde{H _{0} } ^{\left( 2 \right) } \left( p _{1} , p _{2} \right) = G \left( p _{1} , p _{2} \right) g \left( p _{1} , p _{2} \right) $ with $ G \left( p _{1} , p _{2} \right) = p _{1} ^2 + p _{2} ^2 + \frac{2}{m+1} \left( p _{1} , p _{2} \right) $ for $ g \in L _{2} ^{\text{asym} } \left( \mathbb{R}^{3} \times \mathbb{R}^{3} \right) $ s.t. $ \int_{\mathbb{R}^{3} } g \left( p _{1} , p _{2} \right) dp _{j} = 0. $

\medskip

\begin{thebibliography}{1}
\bibitem{griffiths}
Griffiths, David J. \textit{Introduction to Elementary Particles}. Weinheim: Wiley-VCH, 2008.

\end{thebibliography}

\end{document}
