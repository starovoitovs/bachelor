\documentclass[11pt, a4paper, german]{article}
\usepackage{amsmath,amsthm,amssymb,latexsym,amsfonts,breqn}
\usepackage[left=3cm,right=3cm,top=2	cm,bottom=2cm]{geometry} % page settings
\setlength{\parindent}{0mm}

\usepackage[ngerman]{babel}
\usepackage[utf8]{inputenc}
\usepackage{BA_Titelseite}

\theoremstyle{plain}
\newtheorem{theorem}{Theorem}

\theoremstyle{definition}
\newtheorem{definition}[theorem]{Definition}
\newtheorem{example}[theorem]{Example}

\theoremstyle{remark}
\newtheorem{remark}[theorem]{Remark}

\numberwithin{equation}{section}
\numberwithin{theorem}{section}

%Namen des Verfassers der Arbeit
\author{Konstantins Starovoitovs}
%Geburtsdatum des Verfassers
\geburtsdatum{12. Oktober 1992}
%Gebortsort des Verfassers
\geburtsort{Riga, Lettland}
%Datum der Abgabe der Arbeit
\date{\today}

%Name des Betreuers
% z.B.: Prof. Dr. Peter Koepke
\betreuer{Betreuer: Prof. Dr. Margherita Disertori}
%Name des Instituts an dem der Betreuer der Arbeit tätig ist.
\zweitgutachter{Zweitgutachter: Prof. Dr. Juan J. L. Velázquez}
%z.B.: Mathematisches Institut
\institut{Institute for Applied Mathematics}
%Titel der Bachelorarbeit
\title{Extensions of symmetric operators in the context of point-particle interaction}
%Do not change!
\ausarbeitungstyp{Bachelorarbeit Mathematik}

\DeclareMathOperator{\Exists}{\exists}
\DeclareMathOperator{\Forall}{\forall}

\begin{document}

\maketitle

%\clearpage

\begin{abstract}
In dieser Bachelorarbeit machen wir eine Zusammenfassung der Theorie selbstadjungierter Erweiterungen symmetrischer Operatoren und zeichnen eine spezielle Klasse der Erweiterungen aus: Ter-Martirosian--Skornyakov Erweiterungen. Des Weiteren zeigen wir eine Anwendung dieser Erweiterungen im Kontext von Partikel-Interaktion.\\
Im ersten Kapitel machen wir eine kurze Wiederholung der Begriffe aus der Quantenmechanik. Im zweiten Kapitel motivieren wir die Nutzung der unbeschränkten Operatoren und beschreiben deren Eigenschaften. Im dritten Kapitel wir wenden uns der Frage der Erweiterungen symmetrischer Operatoren. Im vierten Kapitel wenden wir die Theorie im Kontext  von Partikel-Interaktion an.
\end{abstract}

\section{Particle kinematics}

Before we dive into theory of symmetric operators and their extensions, we want to describe model of particle physics based on \cite{griffiths}.\\

Unlike in Newtonian mechanics, where state of the system is given by the vectors in Euclidean space $\mathbb{R} ^d$ and physical quantities by functions on this space, states in quantum mechanics are represented by functions in a Hilbert space $\mathcal{H} = L_2(\mathbb{R} ^d)$ and physical quantities by self-adjoint operators on this space. The functions $\Psi \in \mathcal{H}$ are called wave functions.\\

In fact functions that are multiples of one another represent the same quantum state (so basically we could say that the state is represented by an equivalence class of wave functions). It is customary to normalize wave functions so that $\int |\Psi|^2 d^3x=1$. Then wave functions stand for probability amplitude of the state (meaning that $|\Psi |^2$ is probability distribution, i.e. $|\Psi(x,y,z,t)|^2 \, dx \, dy\,dz$ is the probability of finding the the particle in the volume element $dx\,dy\,dz$).\\

Quantum mechanics postulates that orbital angular momentum $L$ and spin angular momentum $S$ of a particle are given by the formulas:
\begin{align}
L^2 &= l(l+1)\hbar^2 &l   &\in \mathbb \{0, 1, 2, \dots\} \label{eq:l2} \\
L_z &= m_l\hbar      &m_l &\in \{-l, -l+1, \dots, l-1, l\} \label{eq:lz}\\
S^2 &= s(s+1)\hbar^2 &s   &\in {0, 1/2, 1, 3/2, \dots} \label{eq:s2}\\
S_z &= m_s\hbar      &m_s &\in \{-s, -s+1, \dots, s-1, s\} \label{eq:sz}
\end{align}

Notice that we can only measure one coordinate of momentum ($z$ by convention) and its modulus squared due to Heisenberg's uncertainty principle.

\begin{definition}
Particles of half-integer spin are called \textbf{fermions}, particle of integer spin are called \textbf{bosons}.
\end{definition}

An important difference between fermions and bosons is that two identical bosons can occupy the same quantum space, whereas fermions cannot (Pauli exclusion principle). This accounts for the fact that two identical bosons have symmetrical wave function ($\psi(x, y) = \psi(y, x)$) and fermions have antisymmetrical wave function ($\psi(x, y) = -\psi(y, x)$). Therefore, if we assume for two fermionic particles that $x=y$, then $\psi(x,y)=\psi(y,x)=0$, which is consistent with Pauli exclusion principle.\\

\section{Unbounded operators}\label{sec:self-adjoint-operators}

As mentioned before, physical quantities in quantum mechanics are given by self-adjoint operators. The model described in the previous chapter assumes existence of a well-defined Hamiltonian. However, the Hamiltonian defined in (\ref{eq:general-hamiltonian}) is not necessarily self-adjoint, so in order to get a well-defined problem we have to adjust the operator $H$. This is where extensions of symmetric operators come into play.\\

We are used to deal with well-behaved bounded operators (meaning $\Vert A\Vert = \sup_{\Vert \psi \Vert = 1} \Vert A \psi \Vert < \infty$). However, in physics we often encounter operators which are not bounded. Before we proceed, first we have to define how theory of bounded operators extrapolates to the unbounded operators. \\

We start with some separable Hilbert space $\mathcal{H}$. We notice that $L_2(\mathbb R ^n)$ is separable.

\begin{definition} The linear operator $T$ acting on the space $\mathcal{H}$ is called \textbf{bounded}, if $$\sup_{\Vert x \Vert = 1}  \Vert Tx \Vert < \infty.$$ If a linear operator is not bounded, we call it \textbf{unbounded}. We denote with $D(T)$ the domain of the operator $T$.
\end{definition}

\begin{definition}
The \textbf{graph} of the linear operator $T$ is defined by $$\Gamma(T) := \{(\varphi, T\varphi): \varphi \in D(T)\}.$$ We consider the graph $\Gamma(T)$ as the subspace of the Hilbert space $\mathcal{H}\times \mathcal{H}$ equipped with the scalar product $$((\varphi_1, \psi_1), (\varphi_2, \psi_2))_{\mathcal{H}\times\mathcal{H}} := (\varphi_1, \varphi_2)_{\mathcal{H}} + (\psi_1, \psi_2)_{\mathcal{H}}.$$
\end{definition}

\begin{definition}
Let $T_1$ and $T_2$ be operators on $\mathcal{H}$. We say that $T_2$ is an \textbf{extension} of $T_1$, if $$\Gamma(T_1) \subset \Gamma(T_2).$$
\end{definition}

We observe that the definition implies that $D(T_1)\subset D(T_2)$ and $T_2|_{D(T_1)} = T_1$.

\begin{definition}
An operator $T$ is called \textbf{closed}, if its graph $\Gamma(T)$ is closed in $\mathcal{H}\times\mathcal{H}$. An operator $T$ is called \textbf{closable} if it has a closed extension. Every closable operator has the smallest closed extension, called its \textbf{closure}, which we denote by $\overline{T}$.
\end{definition}

\begin{remark}
$\overline{\Gamma(T)}$ is not necessarily closed.
\end{remark}

Closed operators constitute a class of operators that are not necessarily continuous, but still retain nice enough properties such that, for example, one can define the spectrum.

\begin{example}
Let $T$ be the derivative operator $T : C([a,b]) \rightarrow C([a,b]):= \frac{d}{dx}$ (where $C([a,b])$ is considered as Banach space with supremum norm). If we take $D(T):=C^1([a,b])$, then $\text{Ran}(T)=C([a,b])$ and $T$ is closed operator (due to fundamental theorem of calculus). But if we take $D(T):=C^\infty([a,b])$, then $\text{Ran}(T)=C^\infty([a,b])$ and $T$ is not closed (since $C^\infty$ is dense in $C^k$) but has a closure which is its extension to $C^1([a,b])$ (and hence closable).
\end{example}

\begin{definition}
Let $T$ be a densely defined linear operator on a Hilbert space $\mathcal{H}$. We denote with $$D(T^*) := \{\varphi \in \mathcal{H} : \Exists \eta \in \mathcal{H} \Forall \psi \in D(T) (T\psi, \varphi) = (\psi, \eta) \}.$$ For each such $\varphi \in D(T^*)$ we define $T^*\varphi := \eta$. The operator $T^*$ is called the \textbf{adjoint} of $T$.
\end{definition}

We observe that by the Riesz lemma $\varphi \in D(T^*)$ if and only if $|(T\psi, \varphi)|\leq C\Vert\psi\Vert$ (since the operator $(T \bullet, \varphi)$ must be bounded on the domain $D(T)$).\\

We also infer directly from the definition that $S\subset T$ implies $T^*\subset S^*$.

\begin{theorem}
Let $T$ be a densely defined operator on a Hilbert space $\mathcal{H}$. Then the following holds:
\begin{enumerate}
\item $T^*$ is closed.
\item $T$ is closable if and only if $D(T*)$ is dense in which case $\overline T = T ^{**}$.
\item If $T$ is closable, then $\overline T ^* = T ^*$.
\end{enumerate}
\end{theorem}

\begin{definition}
A densely defined linear operator $T$ on the Hilbert space $\mathcal{H}$ is called \textbf{symmetric} if $T\subset T^*$. Equivalently, $T$ is symmetric if and only if $$(T\varphi, \psi) = (\varphi, T\psi) \Forall \psi, \varphi \in D(T).$$
\end{definition}

We observe that symmetric operator is always closable since $D(T)\subset D(T^*)$ and $T^*$ is closed.

\begin{definition}
$T$ is called \textbf{self-adjoint} if $T = T^*$, i.e. if and only if $T$ is symmetric and $D(T) = D(T^*)$. A symmetric operator $T$ is called \textbf{essentially self-adjoint} if $T^*$ is self-adjoint.
\end{definition}

One immediate consequence is that the spectrum of the self-adjoint operator is real: let $\alpha \in \mathbb C $ such that $\alpha \varphi = T \varphi$ and $\alpha \neq 0$ and $\varphi \neq 0$. Then:

\begin{equation}
\overline \alpha = \overline \alpha \frac \alpha \alpha = \alpha \frac{(T\varphi, \varphi)}{(\varphi, T\varphi)} = \alpha \frac{(\varphi, T\varphi)}{(\varphi, T\varphi)} = \alpha
\end{equation}

We notice that the spectrum of a self-adjoint operator is real, which is why they are of tremendous importance in the first place (since the observables in quantum mechanics are real). Let $T$ be self-adjoint. Consider $\varphi \in D(T)$ such that $T\varphi=i\varphi$ an eigenvector for $i$. Then

\begin{equation}
-i(\varphi, \varphi) = (i\varphi,\varphi) = (T\varphi,\varphi) = (\varphi, T\varphi) = i(\varphi,\varphi)
\end{equation}

which implies $\varphi = 0$. Same applies to $-i$. This suggests that $T^*\pm i$ might play an important role. We actually have the following result

\begin{theorem} (Basic criterion of self-adjointness) Let $T$ be a symmetric densely defined operator. The following statements are equivalent:

\begin{enumerate}
\item $T$ is essentially self-adjoint
\item $T$ is $\ker(T^*\pm i) = 0$
\item $\text{Ran}(T\pm i)$ are dense
\end{enumerate}

\end{theorem}

Important fact is that the spectral theorem holds, which means that self-adjoint operators are multiplication operators up to some unitary transformation.

\begin{theorem}
Let $A$ be a self-adjoint operator on a separable Hilbert space $\mathcal H$ with domain $D(A)$. Then there is a measure space $(M, \mu)$ with a finite measure $\mu$, a unitary operator $U: \mathcal H \rightarrow L^2(M, d\mu)$ and a real-valued function $f$ on $M$ which is finite a.e. so that:

\begin{itemize}
\item $\psi \in D(A)$ if and only if $f(\cdot)(U\psi)(\cdot)\in L^2(M, d\mu)$;
\item If $\varphi \in U(D(A))$, then $(UAU^{-1} \varphi)(m) = f(m)\varphi(m)$.
\end{itemize}
\end{theorem}

We recall that the time-evolution is a system is determined by the wave function 

\begin{equation}
\Phi(x,y,z,t) = \phi(x,y,z)\exp(-itH)
\end{equation}\\

Another fundamental result is:

\begin{theorem} (Hellinger--Toeplitz)
Everywhere defined self-adjoint operator is bounded.
\end{theorem}

The fact that everywhere defined self-adjoint operators are bounded suggests that unbounded self-adjoint operators are defined not everywhere, but rather on a dense subspace of $H$.

\begin{example}

Let $d=1$ and $(T \varphi)(x) := x \varphi (x)$ (position operator). We notice that $T \varphi \in \mathcal{H}$ if and only if $\int x^2 |\varphi(x)|^2 < \infty$, so we define $D(T):= \{\varphi \in \mathcal{H} :\int x^2 |\varphi(x)|^2 < \infty \}$. We notice that the operator is unbounded (i.e. we can find a sequence $\Vert T \varphi_n \Vert \rightarrow \infty$ while keeping $\Vert \varphi_n \Vert = 1$, consider, for example, standard mollifiers). So $T$ is an example of an unbounded self-adjoint operator defined on a dense subset of $\mathcal{H}$.

\end{example}

The example above suggests that in physics we usually come across unbounded operators, which highlights the importance of the theory of self-adjoint extensions of symmetric operators.

\section{Self-adjoint extensions}\label{sec:self-adjoint-extensions}

The operators that we come across in the context of point-particle interaction are usually unbounded, and in general not self-adjoint. However, observables in quantum mechanics must be given by self-adjoint operators. Therefore we have to look for self-adjoint extensions of the initial operator, each such extension accounts for different behavior of the system.\\

The question of the self-adjointness of an extension usually cannot be answered directly and can depend on:

\begin{itemize}
\item number of the particles ($N=2, 3, \dots$);
\item dimension of the space ($d=1,2,3$);
\item ratio of the masses of the particles;
\item symmetries of the space:
\begin{itemize}
\item bosonic (symmetry) or fermionic (antisymmetry) cases;
\item translational symmetries;
\item rotational symmetries.
\end{itemize}
\end{itemize}

Each configuration has to be studied separately and the results may vary substantially (for example, there is large difference between the models of 2 and models of 3 particles).\\

There are different approaches that allow us to identify self-adjoint extensions of a symmetric operator. First of all, an extension can be viewed as a perturbation of the initial operator (\textit{free} Hamiltonian). However, this approach has only limited application (for example in case $N=2, d=1$).\\

More complicated configurations make use of more elaborate theories of extensions of symmetric operators. We are going to present the results of two parallel theories that identify self-adjoint extensions of symmetric operators using some auxiliary constructs: Krein-Višik-Birman (KVB) and von Neumann theories of extensions of symmetric operators.

\subsection{Friedrichs extension}

There is a distinguished extension of a semi-bounded operator $T$. As stated above, if a semi-bounded operator has equal deficiency indices, then the operator is essentially self-adjoint. There is a distinguished extension called \textbf{Friedrichs} extension. The existence of this extension will later prove to be very useful when developing KVB theory of self-adjoint extensions. We now want to construct this extension.

\begin{definition}
Consider $q(\varphi, \psi) := (\varphi, A\psi)$ with $\varphi, \psi \in D(A)$. Then $q$ is a closable quadratic form and its closure $\widehat{q}$ is the quadratic form of the unique operator $\widehat{A}$. Then $\widehat{A}$ is a positive extension of $A$, and the lower bound of its spectrum is the lower bound of $q$ (hence we preserve semi-boundedness). Further, $\widehat{A}$ is the only self-adjoint extension of $A$ whose domain is contained in the form domain $\widehat{q}$. We call this extension Friedrichs extension.
\end{definition}

As we stated above, two properties of the Friedrichs extension is that we preserve the lower bound and that the domain of $\widehat{A}$ is contained in the form domain of $\widehat{q}$. Moreover, we have the following result:

\begin{theorem}
Let $A$ be a symmetric operator which is bounded from below. If the Friedrichs extension is the only self-adjoint extension of $A$ that is bounded from below, then the operator is essentially self-adjoint.
\end{theorem}

So we can summarize the following result about strictly positive symmetric operators:

\begin{theorem}

Let $A$ be strictly positive symmetric operator, i.e. $(A\varphi, \varphi) \geq c(\varphi, \varphi)$ for all $\varphi \in D(A)$ and some $c > 0$. Then the following are equivalent:

\begin{itemize}
\item $A$ is essentially self-adjoint
\item $\text{Ran}(A)$ is dense
\item $\ker	A^* = 0$
\item $A$ has only one semi-bounded self-adjoint extension.
\end{itemize}

\end{theorem}


\subsection{Von Neumann theory of self-adjoint extensions}

Von Neumann theory of self-adjoint extensions is based on classifying self-adjoint extensions of a symmetric operator using the notion of the deficiency spaces and deficiency indices. The crowning achievement of the theory is the criterion for the existence of the self-adjoint extensions of a symmetric operator and one-to-one correspondence between the self-adjoint extensions and isometries of deficiency spaces.

\begin{theorem}
Let $A$ be closed symmetric operator. Then the following holds:
\begin{itemize}
\item $\dim \ker (\lambda I - A^*)$ is constant throughout the open upper half-plane
\item $\dim \ker (\lambda I - A^*)$ is constant throughout the open lower half-plane
\item The spectrum of $A$ is one of the following: 
	\begin{itemize}
	\item The closed upper half-plane
	\item The closed lower half-plane
	\item The entire plane
	\item A subset of real axis
	\end{itemize}
\item $A$ is self-adjoint if and only if spectrum is subset of real axis
\item $A$ is self-adjoint if and only if the dimensions of the spaces above are zero.
\end{itemize}
\end{theorem}

This theorem indicates that $\ker (i - A^*)$ and $\ker (i + A^*)$ play an important role. We make the following definitions:

\begin{definition}
$$\mathcal{K}_+:=\ker (i-A^*) = \text{Ran}(i+A)^{\perp}$$
$$\mathcal{K}_-:=\ker (i+A^*) = \text{Ran}(i-A)^{\perp}$$
are called \textbf{deficiency subspaces} of $A$. The dimensions of these spaces are called the \textbf{deficiency indices} of $A$.

\end{definition}

We introduce the following sesquilinear forms:

$$(\varphi, \psi)_A = (\varphi, \psi) + (A^*\varphi, A^*\psi)$$
$$[\varphi, \psi]_A = (A^*\varphi, \psi) - (\varphi, A^*\psi)$$

We call a subspace of $D(A^*)$ $A$-symmetric, if $[\varphi, \psi]_A = 0$ for all $\varphi, \psi$ in the subspace. We also define $A$-closedness and $A$-symmetricity with respect to $(\cdot, \cdot)_A$. The following result holds:

\begin{theorem}\label{von-neumann-dec}
Let $A$ be a closed symmetric operator. Then:
\begin{itemize}
\item The closed symmetric extensions of $A$ are the restrictions of $A^*$ to $A$-closed, $A$-symmetric subspaces of $D(A^*)$.
\item $D(A^*) = D(A)\oplus_A \mathcal{K}_+ \oplus_A \mathcal{K}_-$
\item There is one-to-one correspondence between $A$-closed, $A$-symmetric subspaces $S$ of $D(A^*)$ which contain $D(A)$ and the $A$-closed, $A$-symmetric subspaces $S_1$ of $\mathcal{K}_+\oplus_A\mathcal{K}_-$ given by $S_1 = D(A) \oplus_A S_1$.
\end{itemize}
\end{theorem}

The main theorem is the following:

\begin{theorem}
Let $A$ be a closed  symmetric operator. The closed symmetric extensions of $A$ are in one-to-one correspondence with the set of partial isometries (in the usual inner product) of $\mathcal{K}_+$ into $\mathcal{K}_-$. If $U$ is such an isometry with the initial space $I(U) \subset \mathcal{K} _+$, then the corresponding closed symmetric extension $A_U$ has domain:

\begin{equation}
D(A_U)=\{\varphi + \varphi_+ + U \varphi_+ | \varphi \in D(A), \varphi_+ \in I(U)\}
\end{equation}

\end{theorem}

The theorem has the following consequences:

\begin{itemize}
\item $A$ is self-adjoint if and only if $n_+ = 0 = n_-$
\item $A$ has self-adjoint extensions if and only if $n_+ = n_-$. There is one-to-one correspondence between self-adjoint extensions of $A$ and unitary maps from $\mathcal{K}_+$ onto $\mathcal{K}_-$
\item If either $n_+ = 0 \neq n_-$ or $n_- = 0 \neq n_+$, then $A$ has no non-trivial symmetric extensions.
\end{itemize}

Another useful observation is the following:    

\begin{definition}
Anti-linear map is called \textbf{conjugation} if it is norm-preserving and $C^2 = I$.
\end{definition}

\begin{theorem}
If a symmetric operator $A$ commutes with some conjugation $C$, then it has equal deficiency indices and therefore has self-adjoint extensions.
\end{theorem}

\subsection{KVB (Krein-Višik-Birman) theory of self-adjoint extensions}

We denote with

$$m(S) := \inf _{f\in D(S)\setminus \{0\}}\frac{(f, Sf)}{\Vert f \Vert ^2}$$ 

the \textbf{bottom} of a semi-bounded symmetric operator $S$, i.e. its greatest lower bound. We assume that $S$ is densely-defined. We recall that, if the operator is symmetric, it is always closable, so we can always refer to the closure of a given operator. First of all, we notice that the theory does not impose restrictions on the deficiency indices of $S$ ($\dim \ker (S^* \pm i)$), in general they can be infinite. Moreover, via translation we can assume that $m(S) > 0$, so that we can develop theory for strictly positive operators without loss of generality (consider $S_\lambda := S + \lambda I$ and correspondingly translate the results to the initial operator). Hence we can assume the existence of the Friedrichs extension $S_F$ with bounded inverse.\\

As in the von Neumann theory of self-adjoint extensions, we are first of all interested in the definition of $D(S^*)$, i.e. the decomposition with respect to the corresponding deficiency spaces. Within the von Neumann theory the following relation holds:

$$D(S^*) = D(S) + \ker(i+S^*) + \ker(i-S^*)$$

i.e. a direct sum of the initial domain and both deficiency spaces, this formula is valid for any densely defined $S$.\\

Unlike von Neumann theory, the KVB theory works with the real version of the equation above, i.e. we examine $ \ker S^* $ instead of the two deficiency spaces. We refer to $\ker S^*$ as \textit{the} deficiency space.

\begin{theorem}\label{kvb-dec}
For a densely defined $S$ we have: 

\begin{align}
D(S^*) &= D(S_F) + \ker S^*\\
D(S^*) &= D(\overline S) + S_F^{-1} \ker S^* + \ker S^*\\
D(S_F) &= D(\overline S) + S_F^{-1} \ker S^*
\end{align}

\end{theorem}

The following theorem classifies self-adjoint extensions of a given operator:

\begin{theorem}\label{kvb-main}
Let $S$ be a densely-defined symmetric operator with positive bottom. There is one-to-one correspondence between the family of all self-adjoint extensions of $S$ and the family of the self-adjoint operators on Hilbert subspaces of $\ker S^*$. If $T$ is any such operator, in the correspondence $T \leftrightarrow S_T$ each self-adjoint extension $S_T$ of $S$ is given by

\begin{align}
S_T &= S^* \upharpoonright D(S_T) \\
D(S_T) &= \left\{ f + S_F^{-1} (Tv + w) = v \middle| f\in D(\overline S), v \in D(T), w \in \ker S^* \cap D(T)^{\perp} \right\}
\end{align}

\end{theorem}

KVB theory also allows to make a statement about the bound of the bottom of a self-adjoint extension based on the bottom of the initial operator.

\begin{theorem}\label{kvb-est}
Let $S$ be a densely defined symmetric operator with positive bottom. If $S_T$ is a self-adjoint extension of $S$ and if $\alpha < m(S)$, then the following are equivalent:

\begin{align}
(g, S_T g) &\geq \alpha \Vert g \Vert ^2 \,\,\,\,\,\forall g \in D(T)\\
(v, Tv) &\geq \alpha \Vert v \Vert ^2 + \alpha ^2 (v, S_F-\alpha I ) ^{-1} v) \,\,\,\,\, \forall v \in D(T)
\end{align}

It follows that $m(T)\geq m(S_T)$ for any semi-bounded $S_T$. In particular:

\begin{align}
m(S_T)\geq 0 \,\Leftrightarrow\, m(T) \geq 0 \\
m(S_T)> 0 \,\Leftrightarrow\, m(T) > 0
\end{align}

\end{theorem}

The Friedrichs extension $S_F$ is uniquely identified by specific choice of the operator $T$:

\begin{theorem}
Let $S$ be a densely-defined symmetric operator with positive bottom and let $S_T$ be a positive self-adjoint extension of $S$ parametrised by $T$. Then $S_T$ is the Friedrichs extension when $D(T)=\{0\}$.
\end{theorem}

As we mentioned before, KVB theory allows for the deficiency index $\dim \ker S^*$ to be infinite. In case $\dim\ker S^* < \infty$, the situation is somewhat simpler:

\begin{theorem}
Let $S$ be a densely-defined symmetric operator with positive bottom and finite deficiency index ($\dim\ker S^*< \infty$). Then:
\begin{itemize}
\item the semi-boundedness of $S_T$ is equivalent to the semi-boundedness of $T$;
\item any self-adjoint extension of $S$ is bounded from below.
\end{itemize}
\end{theorem}

\section{Extensions of symmetric operators in the context of particle interaction}

After summarizing theoretical basis required for the study of the extensions of symmetric operators, now we want to apply the results to study the particle interaction.

\subsection{One particle system in the presence of central potential}

The principal foundation for non-relativistic quantum mechanics is Schrödinger equation. The law of conservation of energy in quantum mechanics signifies that if in a given state the energy (sum of the kinetic and potential energy) has a definite value, this value remains constant in time:

\begin{equation} \label{eq:energy}
\frac{1}{2m} p^2 + V = E
\end{equation}

State in which the energy has definite values are called \textbf{stationary states} of a system. The stationary state with the smallest possible value of the energy is called \textbf{ground state}.\\

The expression above can be rewritten as 

\begin{equation} \label{eq:schroedinger}
\left(- \frac{\hbar^2}{2m}\Delta + V\right)\Psi = i\hbar \frac{\partial}{\partial t} \Psi
\end{equation}

where $\Psi(x,y,z,t)$ is our wave function.\\

Notice that the Schrödinger equation applies for the cases when the constituents of the system travel at speeds substantially less than $c$ (hydrogen, hadrons made out of heavy quarks and others, as opposed to, for example, photon), so we can use non-relativistic quantum mechanics.\\

This is the time-dependent Schrödinger equation which describes the time-evolution of the system of one particle of mass $m$ in the presence of a specified potential energy $V$.\\

In case $\Psi$ is self-adjoint, we can apply spectral theorem and can write

\begin{equation} \label{eq:wave-function-separation}
\Psi(x,y,z,t) = \psi(x,y,z) f(t)
\end{equation}

By separation of variables we get 

\begin{equation} \label{eq:schroedinger-after-separation}
\frac{1}{\psi}\left(- \frac{\hbar^2}{2m}\Delta + V\right)\psi = \frac{i\hbar}{f}\frac{df}{dt}
\end{equation}

The only way this equation can hold for all $x,y,z$ and $t$ is if both sides are constant:

\begin{align} 
\left(- \frac{\hbar^2}{2m}\Delta + V\right)\psi = E \psi \label{eq:psi}\\
i\hbar \left(\cfrac{\partial f}{\partial t} \right) = Ef \label{eq:f}
\end{align}

We can easily solve (\ref{eq:f}) and get the following formula for the time-evolution of the energy of the system: 

\begin{equation} \label{eq:time-evolution}
f(t) = \exp(-itE/\hbar)
\end{equation}

The operator $H:= - \frac{\hbar^2}{2m}\Delta + V$ is called \textbf{Hamiltonian} and corresponds to the total energy of the system. (\ref{eq:psi}) has eigenvalue form: 

\begin{equation} \label{eq:schroedinger-eigenvalue-form}
H\psi = E\psi
\end{equation}

So after we find $\psi$ which satisfies the first equation, the complete wave function for a particle of mass $m$ and energy $E$, under the influence of potential energy $V(x,y,z)$ is 

\begin{equation}
\Psi(x,y,z,t) = \psi(x,y,z)\exp(-iEt/\hbar)
\end{equation}

It is often the case that the potential $V$ is spherically symmetrical, i.e. depends only on distance from origin $r$ (for example, Coulomb's law).

In this case we can use usual spherical coordinates, in which Laplacian taken the form:

$$ \Delta = \frac{1}{r^2} \frac{\partial}{\partial r}\left(r^2 \frac{\partial}{\partial r}\right) + \frac{1}{r^2 \sin \theta} \frac{\partial}{\partial \theta} \left( \sin \theta \frac{\partial}{\partial \theta}\right) + \frac{1}{r^2 \sin^2 \theta}\frac{\partial ^2}{\partial \phi ^2}$$

By writing $\psi(r, \theta, \phi) = R(r)\Theta(\theta)\Phi(\phi)$, we acquire the following system of ordinary differential equations:

\begin{align} 
\frac{1}{r^2}\frac{d}{dr}\left(r^2 \frac{dR}{dr}\right) &= \left(\frac{l(l+1)}{r^2} + \frac{2m}{\hbar^2}(V(r)-E)\right) R \label{eq:R}\\
\sin \theta \frac{d}{d\theta}\left( \frac{d\Theta}{d \theta}\right) &= \left( m_l^2 - l(l+1)\sin^2\theta\right)\Theta \label{eq:Theta} \\
\frac{d^2\Phi}{d\phi^2} &= -m_l^2\Phi \label{eq:Phi}
\end{align}

where $l$ and $m_l$ are from (\ref{eq:l2}) and (\ref{eq:lz}).\\

It is customary that the solutions of (\ref{eq:Theta}) and (\ref{eq:Phi}) are combined and presented in the form of the spherical harmonics $Y^{m_l}_l(\theta, \phi)$. \\

The equation (\ref{eq:R}) can be simplified by setting $u(r) := r R(r)$ which transforms (\ref{eq:R}) into so-called radial Schrödinger equation:

\begin{equation} \label{eq:radial-schroedinger-equation}
-\frac{\hbar^2}{2m}\frac{d^2u}{dr^2}+\left(V(r)+\frac{\hbar^2}{2m}\frac{l(l+1)}{r^2}\right) = Eu
\end{equation}

At this point the strategy is to solve (\ref{eq:radial-schroedinger-equation}) for particular potential $V(r)$ and combine the result with appropriate spherical harmonic to get the full wave function.

\begin{remark}
For most values of $E$ the solution blows up at large $r$ and yields a non-normalizable wave function. Such solution does not represent a possible physical state. So a bound system cannot have just any old energy, but can take energy on only certain specific values, the so-called \textit{allowed energies} of the system. Our real concern is not with the wave function itself, but with the spectrum of allowed energies.
\end{remark}

\subsection{Derivation of the zero-range potential as the limiting case}

Early developments in the nuclear physics suggested that the nuclear forces interaction must be of very short range and very strong magnitude, which led to the development of the zero-range interaction model. Zero-range interaction is an extremely good approximation of the non-relativistic ultra-cold atom systems, where the effective range of the interaction shrinks to a very small scale \cite{A2}. Here we present motivation behind the model and give mathematical derivation of the potential and behavior of the wave function in the vicinity of the hyperplanes, as presented in \cite{zero-range-potentials}.\\

We denote with $b>0$ the radius of the square-well potential and denote it's energy by $U_0$. Without loss of generality we assume that the potential is centered at the origin, otherwise we can use the Jacobi coordinates ($r \rightarrow r - R$ for the potential centered at $R$). We want to study the system as $b\rightarrow 0$.\\

Let's assume that the wave function $\psi$ is symmetrical, i.e. $\psi = \psi(r)$, and $\psi$ is an eigenfunction for the eigenvalue $E\in \mathbb R$ (binding energy of the system). Then the Schrödinger equation can be rewritten as

\begin{equation}
\frac{1}{r^2}\left[\frac{\partial}{\partial r} (r^2 \frac{\partial}{\partial r} \psi)\right] = 2E \psi
\end{equation}

which reduces to the second order ODE:

\begin{equation}
\psi''+\frac 2r \psi ' - 2E \psi = 0 \label{eq:schrodinger-in-r}
\end{equation}

The equation admits solutions of the form:

\begin{align}
\psi &= \exp(-r\alpha) r^n \\
\psi' &= \exp(-r\alpha) ( n r^{n-1}  - \alpha r^n)\\
\psi'' &= \exp(-r\alpha) (n(n-1) r^{n-2} - 2 \alpha n r^{n-1} + \alpha^2 r^n) )
\end{align}

Plugging back into (\ref{eq:schrodinger-in-r}) yields:

\begin{equation}
\exp(-r\alpha)\left[r^{n-2} (n(n-1) + 2n) + r^{n-1} (-2\alpha n - 2\alpha) + r^n (\alpha^2-2E)\right] = 0
\end{equation}

This equation requires:
\begin{align}
n(n+1)=0 &\Rightarrow n=0 \text{ or } n=1\\
-2\alpha(n+1) = 0 &\Rightarrow n = -1 \text{ or } \alpha = 0\\
-\alpha^2 - 2E = 0 &\Rightarrow \pm \alpha = \sqrt{2E}
\end{align}

These requirements leads us to the following solutions:
\begin{align}
n=0 \quad \alpha=0 \quad E=0 & \quad\quad \Rightarrow \quad\quad \psi= c\label{eq:first-solution}\\[5pt]
n=-1\quad \alpha=0 \quad E=0 & \quad\quad \Rightarrow \quad\quad \psi = \frac{c_1}{r} + c_2\label{eq:second-solution}\\[5pt]
n=-1\quad \alpha\neq0 \quad E \neq 0 & \quad\quad \Rightarrow \quad\quad \psi = \frac{c_1}{r} \exp(-\alpha r) + \frac{c_2}{r} \exp(\alpha r)\label{eq:third-solution}
\end{align}

with $\alpha = \sqrt{2E}$. Since $\exp(\alpha r)$ is not integrable on $\mathbb R ^n \setminus B_b(0)$, $c_2=0$ in (\ref{eq:third-solution}).\\

We also observe that (\ref{eq:third-solution}) is the solution of

\begin{equation}
\frac{\partial}{\partial r}\log r\psi | _{r\rightarrow 0} = -\alpha\label{eq:bc}
\end{equation}

So instead of dealing with the narrow potential with $b>0$, we assume the particle to be free and impose boundary condition (\ref{eq:bc}) on the wave function.\\

Taylor expansion of (\ref{eq:third-solution}) suggests that $\psi = \frac{c}{r}+b$ near 0 (which coincides with (\ref{eq:second-solution})). Plugging this back into (\ref{eq:bc}) yields that

\begin{equation}
b=\alpha c
\end{equation}

is necessary. Hence in the vicinity of 0 the wave function takes the form:

\begin{equation}
\psi(r) = c\left(\frac 1r - \alpha \right) + O(r)\label{eq:tms-condition}
\end{equation}

So we see that zero-range interaction accounts for specific behavior of the wave function near the singularity, i.e. as $r\rightarrow 0$. In the case of several particle interaction we will generalize zero-range interaction by requiring analogous asymptotics near the coincidence hyperplanes, i.e. as $\vert x_i - x_j \vert \rightarrow 0$, for some particle couples $i, j$. Asymptotics of this sort is referred to as the \textbf{TMS condition} \cite{A2}.\\

Noticeably, such asymptotics arise both from physical heuristics for an \textit{effective} low-energy two-body scattering due to an interaction of very short range (as demonstrated above) and from the theory of self-adjoint extensions of symmetric operators, one of the facts that demonstrates strong connection between both radically different approaches.\\

\subsection{Ter-Martirosyan--Skornyakov extensions}

Ter-Martirosyan--Skornyakov (TMS) Hamiltonians are a specific class of extensions of the free Hamiltonian which we will construct in this section.\\

Initially Ter-Martirosyan and Skoryankov exploited the ideas behind Bethe-Peierls theory of deuteron (in particular considering the boundary condition (\ref{eq:bc}) and the potential $U_0 (r) = -(4\pi\hbar ^2 / M)r \delta(r)$ in the Schrödinger equation, which stood for the depth of the well) to derive the integral equation, the solution of which defines the wave functions for the case of three particles. The model is thus parametrized by a single set of parameters $\alpha_{ij}$ as in (\ref{eq:bc}), which corresponds to the scattering lengths in the two-body channel. Following the same approach, similar integral equations were derived for other models with different particle number and symmetries.\\

However, the original equation had solutions for arbitrary values of $-E < 0$. Then the parametrization of the solution was proposed by Danilov, which led to a discrete infinite set of eigenvalues and a family of self-adjoint extensions parametrized by $\{\varepsilon \in \mathbb R\}$, by the way yielding quantitative manifestation of Thomas effect.\\

Minlos and Faddeev proposed the alternative explanation of the parametrization above by placing the problem in the framework of self-adjoint extensions of semi-bounded symmetric operators. Nowadays TMS Hamiltions represent the modern operator-theoretical approach to multi-particle systems with two-body zero-range interaction.\\

TMS Hamiltonians are qualified by two characteristics of acting as the $N$-body $d$-dimensional \textit{free} Hamiltonian on functions that are supported away from the coincidence planes, and of having a domain that consists of square-integrable functions $\Psi(x_1, \dots, x_N)$, possibly with fermionic and bosonic exchange symmetry, which satisfy the aforementioned TMS condition (\ref{eq:tms-condition}) in the vicinity of the hyperplanes ($\vert x_i - x_j\vert \rightarrow 0$) for some or all particle couples $i, j$.\\

In many circumstances, however, the TMS Hamiltonians may fail to be self-adjoint. Moreover, we can stumble upon the following problems that are specific for zero-range interaction:

\begin{itemize}
\item \textbf{Thomas effect}. Infinite discrete sequence of bound states with negative energy diverging to $-\infty$.
\item \textbf{Efimov effect}. Infinite sequence of bound states with negative energy arbitrarily close to zero and non-normalizable eigenfunctions.
\end{itemize}

Properties of the extensions depend on the configuration of a given system, including number of the particles, dimension, space symmetries (fermionic / bosonic case) and particle masses, so that each configuration should be studied separately.\\

Generally we approach the problem as formulated in \cite{A2}:

\begin{enumerate}
\item

Defining the \textit{free} Hamiltonian by restricting it to the regular wave functions that are supported away from the coincidence hyperplanes $\Gamma_{ij}:=\{x_i=x_j\}$. We also embed the bosonic and / or fermionic exchange symmetries (if present) into the space by requiring

\begin{equation}
\psi(\dots, x_i, \dots, x_j, \dots)=\pm \psi(\dots, x_j, \dots, x_i, \dots)
\end{equation}

depending on the symmetry. Rotational symmetries are also possible.

Formally we define the free Hamiltonian as

\begin{equation}
H_0 := - \frac 12 \sum _{i=1} ^N \Delta _{x_n}
\end{equation}

We observe that such Hamiltonian is
\begin{itemize}
\item densely defined (since coincidence planes are null sets);
\item symmetric (with partial integration);
\item positive (since Green's formula yields $\int u(-\Delta u) \text{dx} = \int \vert Du \vert ^2 \text{dx} $);
\end{itemize}

so such Hamiltonian satisfies requirements of von Neumann theory and, if bounded from below, KVB theory presented in the Section \ref{sec:self-adjoint-extensions}, so the study of such Hamiltonian with the frameworks% of these theories can be carried out.

\item 

Since we established that $H_0$ is symmetric and densely defined, we recall from Section \ref{sec:self-adjoint-operators} that any self-adjoint extension $H$ of $H_0$ is the restriction of $H_0^*$, i.e.

\begin{equation}
H_0\subset H \subset H^* \subset H_0^*
\end{equation}

Thus we first characterize $D(H_0^*)$ and then we look for special class of its restrictions based on the TMS condition.

By analogy with (\ref{eq:tms-condition}) we can rewrite the TMS condition as:

\begin{equation}
\Psi(x_1, \dots, x_N) \approx \left( \frac{1}{|x_i - x_j|} - \frac{1}{a_{ij}} \right)
\end{equation}

where $a_{ij}$ stands for scattering length of the interaction with zero range in the $(i,j)$-channel.

\item

After singling out the class of physical extensions labeled $H_\alpha$ by $a=(-a_{ij}^{-1})_{ij}$ we check the extensions for self-adjointness or identify their self-adjoint extensions, and then study their spectral and stability properties.

\end{enumerate}

The last item contains the main problem of checking self-adjointness of extensions. We notice that usually one proceeds by simplifying the problem by reducing it to the equivalent question about self-adjointness of some auxiliary operator (for example, by separating center of mass or by breaking down the operator into trivially self-adjoint and remaining parts, with the latter to be investigated).

\subsection{Two-body point-particle interaction within the KVB theory}

First of all, we observe that we remove the center of mass, i.e. $x:= x_2-x_1$, since the relative part is the one that describes the particle interaction. We notice that in this case the particle interaction is only supported at $x=0$, i.e. whenever $x_2-x_1 = 0$.\\

Our starting operators are:

\begin{equation}
H_0 = -\Delta; \,\,\,\,\, D(H_0) = H^2_0(\mathbb R ^3 \setminus \{0\}),
\end{equation}

where $H^2_0$ denotes the local version of the Sobolev space $H^2$. We know that the Laplacian is a symmetric operator, so we can proceed with the results from the previous chapter.

\begin{remark}
Different papers related to the topic begin the analysis with different spaces, be it the Sobolev space ($H^2_0$), space of infinitely differentiable functions with compact support ($C^\infty_0$) or the Schwartz space of rapidly decreasing functions. However, closures of these spaces with respect to $\Vert \cdot \Vert _{H^2}$ are equal, for example:

\begin{equation}
H^2_0(\mathbb R ^3 \setminus \{0\}) = \overline{C_0^\infty(\mathbb R ^3 \setminus \{0\})} ^ {\Vert \cdot \Vert _{H^2}}
\end{equation}

So we can start by defining the free Hamiltonian on any of these spaces, and then consider the closure of this operator under $\Vert \cdot \Vert _{H^2}$ (we recall that symmetric operators are closable). Hence the initial setups are equivalent.
\end{remark}

We notice that the requirements for KVB theory to be applicable to the operator $H_0$ are trivially met, as pointed out in previous chapter. We follow analysis of self-adjoint extensions from \cite{A2} to illustrate the application of the KVB theory in this context, elaborate proofs can be found in the paper. For convenience we will operate in the Fourier transform.\\

We recall that KVB theory requires us to explicitly calculate the Friedrichs extension of the initial operator, which will in turn allow us to calculate the domain of the adjoint of the operator.

\begin{theorem}
For the operator $H_0$ as given we have:

\begin{align}
D(H_0) &=\left\{f\in H^2(\mathbb R ^3) \middle| \int_{\mathbb R ^3} \widehat{f} (p) \text{dp} = 0 \right\}\\
\widehat{H_0 f}(p) &= p^2\widehat{f}(p)
\end{align}
\begin{align}
D(H_F) &=H^2(\mathbb R^3)\\
\widehat{H_F f}(p) &= p^2\widehat{f}(p)
\end{align}\\

where $H_F$ denotes the Friedrichs extension of $H_0$.

\end{theorem}

The part of the theorem related to the Friedrichs theorem makes use of the fact that it is the only self-adjoint extension whose domain is contained in the form domain of the initial operator (we can consider this a criterion for an extension to be Friedrichs).\\

For the analysis we also need to know $\ker (H^*_0 + \lambda I)$:

\begin{theorem}
For $\lambda > 0$ we have:
\begin{equation}
\ker (H^*_0 + \lambda I) = \left\{ u_\zeta \in L^2(\mathbb R^3) \text{ of the form } \widehat{u}_\zeta (p) = \dfrac{\zeta}{p^2+\lambda} \middle| \zeta \in \mathbb C \right\}
\end{equation}
\end{theorem}

We recall that according to the Theorem \ref{kvb-main} self-adjoint extensions of the operator $H_0$ are in one-to-one correspondence with the self-adjoint operators on Hilbert subspaces of $\ker (H^*_0 + \lambda I)$. The fact that the deficiency space is one-dimensional substantially simplifies the situation. In fact, there are no non-trivial subspaces (only $\{0\}$ and $\ker (H^*_0 + \lambda I)$ itself), and the only self-adjoint maps on the whole space are given by multiplication with $\tau in \mathbb R$, i.e. $T_\tau: u_\zeta \mapsto \tau u _\zeta$, which will account for the family of the self-adjoint extensions.\\

Now in order to determine $D(H^*_0)$ we make use of the Theorem \ref{kvb-dec}:

\begin{theorem}
Adjoint of the operator $H_0$ as above is given by:
\begin{align*}
D(H_0^*) &= \left\{g\in L^2(\mathbb R ^3) \middle| \widehat g (p) = \widehat f (p) + \dfrac{\eta}{(p^2+\lambda)^2} + \dfrac{\zeta}{p^2 + \lambda}; \,\,\,\,\, f \in D(H_0); \,\,\,\,\, \eta, \zeta \in \mathbb C\right\}\\
\widehat{H_0^* h}(p) &= p^2\widehat g(p) - \zeta
\end{align*}
\end{theorem}

Now with the main Theorem \ref{kvb-main} we can identify the whole family of self-adjoint extensions of $H_0$:

\begin{theorem}
The self-adjoint extensions of the operator $H_0$ are given by the family $\{H_0^{(\tau)}|\tau \in \mathbb R \cup \{\infty\}\}$, where $H_0^{(\infty)}$ is the Friedrichs extension and

\begin{align*}
D(H^{(\tau)}_0)&=\left\{g \in L^2(\mathbb R ^3) \middle| \widehat g (p) = \widehat f (p) + \dfrac{\tau \zeta}{(p^2+\lambda)^2} + \dfrac{\zeta}{p^2 + \lambda}; \,\,\,\,\, \zeta \in \mathbb C; \,\,\,\,\, f \in D(H_0)\right\}\\
\widehat{H^{(\tau)}_0 g}(p) &= p^2\widehat g (p) - \zeta
\end{align*}
\end{theorem}

The estimate in the Theorem \ref{kvb-est} allows us to make a statement about the bottom of an extension:

\begin{theorem}
Each extension $H^{(\tau)}_0$ is semi-bounded from below. Moreover:
\begin{align}
m(H^{(\tau)}_0) \geq 0 \Leftrightarrow \tau \geq 0 \\
m(H^{(\tau)}_0) > 0 \Leftrightarrow \tau > 0
\end{align}\\
\end{theorem}

\begin{remark}
To obtain the family of the self-adjoint extensions, we made use of the KVB theory. Alternatively we could have used von Neumann theory presented in Chapter \ref{sec:self-adjoint-extensions}. Equivalent result is obtained in \cite{solvable-models} Chapter I.1.
\end{remark}

\subsection{TMS Hamiltonians for the three-body problem}

We saw that the three body problem can be treated comparatively easy using the tools from Chapter \ref{sec:self-adjoint-extensions}. The problem that we encounter in the three-body case is that the TMS condition is not self-adjointness condition, hence we have to study self-adjoint extension of the resulting TMS Hamiltonian.\\

Historically the case that drew the most attention was \textit{2+1 fermionic system}, i.e. two indistinguishable fermions and another particle of different nature.\\

First of all, we notice that factor of the Hilbert space that corresponds to the fermions is $ L ^{2} _{\text{asym} } \left( \mathbb{R}^{6} \right) $, so that we get $ \psi \left( x _{1} , x _{2} \right) = - \psi \left( x _{2} , x _{1} \right) $. This yields that $ \psi \left( x_1, x_1 \right) =0 $, i.e. two particles can't be at the same place simultaneously, while still preserving modulus squared (accounting for the fact that the particles are indistinguishable).\\

Hamiltonian of the system in position coordinates is:
\begin{equation}
H _{0} = - \frac{1}{2} \left( \frac{1}{m} \Delta _{y} + \Delta _{x _{1} } + \Delta _{x _{2} } \right)
\end{equation}

After applying Fourier transform, we notice that:

\begin{equation}
\mathcal{F} H _{0} \mathcal{F} ^{-1} \left( \phi \left( q, k _{1} , k _{2} \right) \right) = \left( \frac{1}{m} \left| q \right| ^{2} + \left| k _{1} \right| ^{2} + \left| k _{2} \right| ^{2} \right) \left( \phi \left( q, k _{1} , k _{2} \right) \right)
\end{equation}

In particular we see that operating in Fourier transform allows us to deal with multiplication operator instead of differential operator, which is more convenient.\\

After Fourier transform (going to momentum space) we make the change of coordinates:

\begin{equation}
P = q+k _{1} + k _{2}; \,\,\,\,\, p _{j} = \frac{P }{m+2} - k _{j}
\end{equation}

Corresponding matrices are:

\begin{equation}
U = \left( \begin{matrix} 1 & 1 & 1 \\[10pt] \dfrac{1}{m+2} & \dfrac{1}{m+2} -1 & \dfrac{1}{m+2} \\[10pt] \dfrac{1}{m+2} & \dfrac{1}{m+2} & \dfrac{1}{m+2} -1 \end{matrix} \right) \,\,\,\, U ^{-1} = \left( \begin{array}{ccc} 1 - \dfrac{2}{m+2} & 1 & 1 \\[10pt] \dfrac{1}{m+2} & -1 & 0 \\[10pt] \dfrac{1}{m+2} & 0 & -1 \end{array} \right) 
\end{equation}\\

So we get:

\begin{align}
q      &= \left(1 - \frac{2}{m+2} \right) P + p _{1} + p _{2} \\
k _{1} &= \frac{P}{m+2} - p _{1} \\
k _{2} &= \frac{P}{m+2} - p _{2}
\end{align}

which yields:

\begin{align*}
\dfrac{1}{m} \left| q \right| ^{2} + \left| k _{1} \right| ^{2} + \left| k _{2} \right| ^{2} &=
\dfrac{P ^{2} }{m} - \dfrac{4 P ^{2} }{m \left( m+2 \right) } + \dfrac{4 P ^{2} }{m \left( m+2 \right) ^2 } + \dfrac{2 P ^2 }{\left( m+2 \right) ^2 } \\&+ \dfrac{2 P p _{1} }{m} - \dfrac{4 P p _{1} }{m \left( m+2 \right) } - \dfrac{2 P p _{1} }{m+2} + \dfrac{2 P p _{2} }{m} - \dfrac{4 P p _{2} }{m \left( m+2 \right) } - \dfrac{2 P p _{2} }{m+2} + \dfrac{p _{1} ^2 }{m} \\&+ \dfrac{2 \left( p _{1} , p _{2} \right) }{m} + \dfrac{p _{2} ^2 }{m} + p _{1} ^2 + p _{2} ^2 \\&= \dfrac{P ^{2} }{m} - \dfrac{4 P ^{2} }{m \left( m+2 \right) } + \dfrac{4 P ^{2} }{m \left( m+2 \right) ^2 } + \dfrac{2 P ^2 }{\left( m+2 \right) ^2 } + \dfrac{p _{1} ^2 }{m} + \dfrac{2 \left( p _{1} , p _{2} \right) }{m} \\&+ \dfrac{p _{2} ^2 }{m} + p _{1} ^2 + p _{2} ^2 \\&= \frac{P ^{2} }{m+2} + \frac{m+1}{m} \left( p _{1} ^2 + p _{2} ^2 + \frac{2}{m+1} \left( p _{1} , p _{2} \right) \right)
\end{align*}\\

Hence we can write: 
\begin{align}
\widetilde{H _{0} } &= \mathcal{F} H _{0} \mathcal{F}^{-1} = \widetilde{H _{0} } ^{\left( 1 \right) } + \frac{m}{m+1} \widetilde{H _{0} } ^{\left( 2 \right) }\\
\left( \widetilde{H _{0} } ^{\left( 1 \right) } f \right) \left( P \right) &= \frac{P ^{2} }{m+2} f \left( P \right)\,\,\,\,\,\,\,\,\,\,\,\,\,\,\,\,\,\,\,\,\,\,\,\,\,\,\,\,\,f\in L _{2} \left( \mathbb{R}^{3} \right)\\
\widetilde{H _{0} } ^{\left( 2 \right) } \left( p _{1} , p _{2} \right) &= G \left( p _{1} , p _{2} \right) g \left( p _{1} , p _{2} \right) \,\,\,\,\,\,\,\,\,\,\,\,g \in L _{2} ^{\text{asym} } \left( \mathbb{R}^{3} \times \mathbb{R}^{3} \right)\\
G \left( p _{1} , p _{2} \right) &= p _{1} ^2 + p _{2} ^2 + \frac{2}{m+1} \left( p _{1} , p _{2} \right) \\
\end{align}

provided

\begin{equation*}
\int_{\mathbb{R}^{3} } g \left( p _{1} , p _{2} \right) dp _{j} = 0
\end{equation*}

Therefore we can reduce the analysis to the study of the operator $\widetilde{H _{0} } ^{\left( 2 \right) } $, since $\widetilde{H _{0} } ^{\left( 1 \right) } $ is a self-adjoint operator in $L_2(\mathbb R^3)$. Following the calculations in \cite{B3} (which make use of von Neumann theory of extensions of symmetric operators), we observe that the operator $\widetilde{H _{0} } ^{\left( 2 \right)}$ is symmetric and:

\begin{equation}
D(\widetilde{H _{0} } ^{\left( 2 \right)}) = \left\{g \in L_2^{\text{asym}}(\mathbb R ^3 \times \mathbb R ^3 )\middle| \int _{\mathbb R ^3 }g (p_1, p_2) d p_j, j = 1,2\right\}
\end{equation}

Explicit calculation of the deficiency spaces allows to make use of Theorem \ref{von-neumann-dec} in order to determine the domain and the action of the adjoint. The problem that arises in the three-body problem is that we have wave functions that are zero on the coincidence hyperplanes, therefore an important step in \cite{B3} is to model zero-range interaction by using the domain of the functions that satisfy TMS condition in the vicinity of the coincidence hyperplanes (as opposed to the two-body case, where singularity is in a single point at the origin). It is also demonstrated in \cite{B3} that the problem can be reduced to the study of some auxiliary operator $T$, which in turn can be decomposed into a countable set of operators $M_l$ acting on the domains:

\begin{equation}
D(M_l) = \left\{u\in L_2(\mathbb R_+, r^2 \text{dr})\middle| r u(r) \in L_2(\mathbb R_+, r^2 \text{dr})\right\}
\end{equation}

which are operators on a half-line (spherical part of the decomposition of $L_2(\mathbb R^3)$ was factored out).\\

It should be noted that the three-body problem was also elaborated in \cite{A2} (which we used to depict usage of KVB theory within two-body problem).

\section{Concluding remarks}

We summarized modern methods of the theory of extensions of symmetric operators, motivated the use of the zero-range potentials in the context of point-particle interaction and showed some examples which demonstrated applications to Hamiltonians of systems of several particles.\\

It should be noted that the analysis can differ drastically depending on the configuration of the system. For an in-depth historical reference of the development of the theory refer to \cite{A2}.

\medskip

\begin{thebibliography}{1}
\bibitem{griffiths}
Griffiths, David. \textit{Introduction to Elementary Particles. WILEYVCH.} (2008).
\bibitem{A1}
Michelangeli, Alessandro. \textit{Krein-Vishik-Birman self-adjoint extension theory revisited.} (2015).
\bibitem{A2}
Michelangeli, Alessandro, and Andrea Ottolini. \textit{On point interactions realised as Ter-Martirosyan-Skornyakov Hamiltonians.} arXiv preprint arXiv:1606.05222 (2016).
\bibitem{B3}
Minlos, Robert Adol'fovich. \textit{On pointlike interaction between three particles: two fermions and another particle.} ISRN Mathematical Physics 2012 (2012).
\bibitem{zero-range-potentials}
Demkov, Yu N., and Valentin Nikolaevich Ostrovskii. Zero-range potentials and their applications in atomic physics. Springer Science \& Business Media, 2013.
\bibitem{solvable-models}
Albeverio, Sergio, et al. Solvable models in quantum mechanics. Springer Science and Business Media, 2012.
\end{thebibliography}

\end{document}
